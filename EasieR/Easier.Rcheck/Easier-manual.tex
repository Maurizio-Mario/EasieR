\nonstopmode{}
\documentclass[a4paper]{book}
\usepackage[times,inconsolata,hyper]{Rd}
\usepackage{makeidx}
\usepackage[utf8,latin1]{inputenc}
% \usepackage{graphicx} % @USE GRAPHICX@
\makeindex{}
\begin{document}
\chapter*{}
\begin{center}
{\textbf{\huge Package `Easier'}}
\par\bigskip{\large \today}
\end{center}
\begin{description}
\raggedright{}
\item[Type]\AsIs{Package}
\item[Title]\AsIs{What the Package Does (Title Case)}
\item[Version]\AsIs{0.1}
\item[Date]\AsIs{2016-04-12}
\item[Depends]\AsIs{lattice}
\item[Author]\AsIs{Who wrote it}
\item[Maintainer]\AsIs{Who to complain to }\email{yourfault@somewhere.net}\AsIs{}
\item[Description]\AsIs{More about what it does (maybe more than one line)}
\item[License]\AsIs{What license is it under?}
\item[LazyData]\AsIs{TRUE}
\item[NeedsCompilation]\AsIs{no}
\end{description}
\Rdcontents{\R{} topics documented:}
\inputencoding{utf8}
\HeaderA{z\_plot}{Building a function to translate a score from distribution in a z-score graph.}{z.Rul.plot}
%
\begin{Description}\relax
This function is thought for the chapter 6 on Standard Normal distributions. It allows to rapidly draw a plot with Lattice representing a Standard normal distribution translating the inputs for a generic normal distribution. Furthermore, it allows to stress out cumulative probability distributions if necessary.
\end{Description}
%
\begin{Usage}
\begin{verbatim}
z_plot(x, myMean, mySd, p = 0, title = "", y)
\end{verbatim}
\end{Usage}
%
\begin{Arguments}
\begin{ldescription}
\item[\code{x}] is the score of interest from a given distribution.

\item[\code{myMean}] is the mean of the given distribution.

\item[\code{mySd}] is the standard deviation of the given distribution.

\item[\code{p}] default zero, it is called for plotting and calculating

\item[\code{main}] it should give a desired title to the plot
\end{ldescription}
\end{Arguments}
%
\begin{Details}\relax
the function is actually under development.
\end{Details}
%
\begin{Value}
a plot stressing the corresponding z-score of the input data.
\end{Value}
%
\begin{Author}\relax
Maurizio Mario Murino
\end{Author}
%
\begin{SeeAlso}\relax
\code{Lattice, xyplot}
\end{SeeAlso}
\printindex{}
\end{document}
